\section{INTRODUCCIÓN}

Tradicionalmente la robótica estaba centrada en sectores industriales manufactureros orientados a la producción masiva. A mediados de los 60's se introducen robots manipuladores en distintos tipos de industrias. Típicamente los robots desarrollaban tareas repetitivas, el cual exigía tomar algunas piezas y reubicarlas en otra área a la cual el robot manipulador sea capaz de llegar con la máxima extensión de su articulación lo cual resultaba en un problema. Una solución a este problema fué desarrollar un vehículo móvil sobre rieles y así es como a mediado de los 80's aparecieron los primeros vehículos guiados automáticamente (AGV's).\par
Fuera del entorno industrial, en donde se imposibilitaba estructurar el entorno, se les doto a los robots un mayor grado de inteligencia y capacidad para poder desenvolverse.\par

Uno de los desafíos mas grandes en la aplicación de robots es la navegación en entornos desconocidos abarrotados de obstáculos. La navegación se vuelve aun mas compleja cuando se desconoce la ubicación de estos a priori. Se introdujo así entonces el concepto de conjunto difuso (Fuzzy Set) bajo el que reside la idea de que los elementos sobre los que se construye el pensamiento humano no son números sino etiquetas lingüísticas. \par

La lógica difusa se ha utilizado en el diseño de múltiples posibles soluciones en donde se han creado distintos sistemas de control de navegación orientados a robots para que estos puedan llegar a destino evitando obstáculos en su camino.\par 
A lo largo de los años se han desarrollado algunos enfoques distintos, uno muy particular en el cual la navegación se divide en dos partes. La primera compuesta en comportamientos básicos tales como: lograr metas, evitación de obstáculos y seguimiento de muros. La segunda, una capa de supervisión responsable de la selección de las acciones (elección de comportamientos según el contexto). \par

El principal aporte de mi proyecto de investigación es realizar un sistema de control neuro-difuso nuevo que solo va a necesitar un controlador difuso pero que además, va a estar dividido en dos partes: \par
La primera (difuso) con comportamientos básicos: lograr metas, evitación de obstáculos, etc. A partir de ahora llamado \textbf{"Modulo de sistema de control difuso"}. \par
La segunda de supervisión (neuronal) en donde se va a seleccionar, arbitrar o fusionar comportamientos de la lista de estos en función de la salida del sistema neural convolucional el cual tendrá la capacidad de, a través de una cámara, poder reconocer el obstáculo a evadir y en función de la ubicación y dimensión de este decidir un comportamiento difuso. A partir de ahora llamado \textbf{"Modulo de procesamiento de imágenes"}. \par
Siempre buscando con esta idea que la toma de decisión se parezca un poco mas a la del ser humano, elegir un rumbo en función de lo que el robot ve y reconoce, ya que desde mi punto de vista es antinatural la elección de trayectoria solo en función de la ubicación del obstáculo y el target (destino), hay que tener en cuenta que nosotros, al encontrar un obstáculo que nos evita el paso también evaluamos el trayecto a seguir en función de las dimensiones del obstáculo y del largo del trayecto a recorrer. Por ejemplo al eludir un obstáculo muchas veces evaluamos si eludir por la izquierda o por la derecha es mas conveniente en función del largo de ambos caminos.\par

La detección de objetos es una tarea de suma importancia para la conducción autónoma, junto con esta tarea viene aparejada la responsabilidades de garantizar la precisión a la hora de detectar estos objetos y a su vez, de suma importancia, inferirlo en tiempo real para garantizar el adecuado control del móvil. 
Para satisfacer todas estas necesidades se propuso implementar una SqueezeDet, una red neuronal totalmente convolucional para la detección de objetos caracterizada por su precisión, tamaño, velocidad y consumo de energía. \par

La investigación fue llevada a cabo en el marco de la materia Proyecto Final de la carrera de Ingeniería Electrónica de la Universidad Tecnológica Nacional (Regional Buenos Aires), y fue auspiciada por el Laboratorio de Procesamiento Digital (DPLAB) del Departamento de Electrónica de dicha facultad regional. La investigación realizada y el diseño final quedaron
a disposición del DPLAB como base para futuros trabajos de investigación. \par

Se inicia este trabajo en un capitulo en donde se presenta el marco teorico del proyecto. ... (TERMINAR) \par

